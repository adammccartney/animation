%%%%%%%%%%%%%%%%%%%%%%%%%%%%%%%%%%%%%%%%%
% Stylish Title Page
% LaTeX Template
% Version 2.0 (22/7/17)
%
% This template was downloaded from:
% http://www.LaTeXTemplates.com
%
% Original author:
% Peter Wilson (herries.press@earthlink.net) with modifications by:
% Vel (vel@latextemplates.com)
%
% License:
% CC BY-NC-SA 3.0 (http://creativecommons.org/licenses/by-nc-sa/3.0/)
% 
% This template can be used in one of two ways:
%
% 1) Content can be added at the end of this file just before the \end{document}
% to use this title page as the starting point for your document.
%
% 2) Alternatively, if you already have a document which you wish to add this
% title page to, copy everything between the \begin{document} and
% \end{document} and paste it where you would like the title page in your
% document. You will then need to insert the packages and document 
% configurations into your document carefully making sure you are not loading
% the same package twice and that there are no clashes.
%
%%%%%%%%%%%%%%%%%%%%%%%%%%%%%%%%%%%%%%%%%

%----------------------------------------------------------------------------------------
%	PACKAGES AND OTHER DOCUMENT CONFIGURATIONS
%----------------------------------------------------------------------------------------

\documentclass[a4paper,11pt]{book}
\usepackage{pdfpages}
\usepackage{fancyhdr}
\usepackage{nth}
\usepackage[utf8]{inputenc}
\usepackage{nopageno}
\pagenumbering{gobble}
\usepackage{afterpage}

\setlength{\parindent}{4em}
\setlength{\parskip}{1em}
\renewcommand{\baselinestretch}{1.5}

%booleanclaration
\newboolean{twoside}
\setboolean{twoside}{false}
\usepackage{geometry}
 \geometry{
 a4paper,
 total={170mm,257mm},
 left=20mm,
 top=20mm,
 }


%----------------------------------------------------------------------------------------
%	TITLE PAGE
%----------------------------------------------------------------------------------------

\newcommand*{\titleTH}{\begingroup % Create the command for including the title page in the document
\raggedleft % Right-align all text
\vspace*{\baselineskip} % Whitespace at the top of the page

{\LARGE\bfseries animation}\\[\baselineskip] % First part of the title, if it is unimportant consider making the font size smaller to accentuate the main title

{{\Huge for Flutes, Contrabass Paetzold in F, Electronics, Harp and Harpsichord}}\\[\baselineskip] % Main title which draws the focus of the reader

\vfill % Whitespace between the title block and the publisher

{\large Pangur \\ \small \'Eire - \"Osterreich}  % Publisher and logo

\vspace*{3\baselineskip} % Whitespace at the bottom of the page
\endgroup}

%----------------------------------------------------------------------------------------
%	BLANK DOCUMENT
%----------------------------------------------------------------------------------------

\begin{document} 


\frontmatter
\begin{titlepage}
\begin{center}
\pagestyle{plain} % Removes page numbers
\titleTH % This command includes the title page
\end{center}
\end{titlepage}

\mainmatter
%----------------------------------------------------------------------------------------

\section{Appendix}
\begin{center}
\fancyhf{} % clear all header and footers
\renewcommand{\headrulewidth}{0pt} % remove the header rule
\rfoot{\thepage}
\pagestyle{plain}
\begin{center}
 \includepdf[pages=-, scale=0.9, pagecommand={\thispagestyle{fancy}}]{./animation-appendix.pdf}
    \clearpage\mbox{}\clearpage
\end{center}
\end{center}

%----------------------------------------------------------------------------------------


\section{Score}
\begin{center}
\fancyhf{} % clear all header and footers
\renewcommand{\headrulewidth}{0pt} % remove the header rule
\rfoot{\thepage}
\pagestyle{plain}
\begin{center}

\afterpage{blankpage}
\pagebreak

 \includepdf[pages=-, scale=0.9, pagecommand={\thispagestyle{fancy}}]{./animation.pdf}
\end{center}
\end{center}


%---------------------------------------------------------------------------------------


\section{Parts}
\begin{center}
\fancyhf{} % clear all header and footers
\renewcommand{\headrulewidth}{0pt} % remove the header rule
\rfoot{\thepage}
\pagestyle{plain}

\afterpage{blankpage}
\pagebreak

\subsection{Flutes}
\begin{center}
 \includepdf[pages=-, scale=0.9, pagecommand={\thispagestyle{fancy}}]{./animation_part-flutes.pdf}
\end{center}

\pagebreak

\subsection{Recorders}
\begin{center}
 \includepdf[pages=-, scale=0.9, pagecommand={\thispagestyle{fancy}}]{./animation_part-recorders.pdf}
\end{center}

\pagebreak

\subsection{Synthesizer}
\begin{center}
 \includepdf[pages=-, scale=0.9, pagecommand={\thispagestyle{fancy}}]{./animation_part-synth.pdf}
\end{center}

\pagebreak

\subsection{Harp}
\begin{center}
 \includepdf[pages=-, scale=0.9, pagecommand={\thispagestyle{fancy}}]{./animation_part-harp.pdf}
\pagebreak
\end{center}

\subsection{Harpsichord}
\afterpage{blankpage}
\pagebreak
\begin{center}
 \includepdf[pages=-, scale=0.9, pagecommand={\thispagestyle{fancy}}]{./animation_part-cembalo.pdf}
\end{center}


\end{center}


\end{document}
